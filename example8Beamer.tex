%%%%%%%%%%%%%%%%%%%%%%%%%%%%%%%%%%%%%%%%%%%%%%%%%%%%%%%%%%%%%%%
%
% Welcome to Overleaf --- just edit your LaTeX on the left,
% and we'll compile it for you on the right. If you open the
% 'Share' menu, you can invite other users to edit at the same
% time. See www.overleaf.com/learn for more info. Enjoy!
%
%%%%%%%%%%%%%%%%%%%%%%%%%%%%%%%%%%%%%%%%%%%%%%%%%%%%%%%%%%%%%%%


% Inbuilt themes in beamer
\documentclass{beamer}

% Theme choice:
\usetheme{CambridgeUS}

% Title page details: 
\title{Assignment 3} 
\author{Mahin Bansal}
\date{\today}
\logo{\large \LaTeX{}}


\begin{document}
\newcommand{\myvec}[1]{\ensuremath{\begin{pmatrix}#1\end{pmatrix}}}
% Title page frame
\begin{frame}
    \titlepage 
\end{frame}

% Remove logo from the next slides
\logo{}


% Outline frame
\begin{frame}{Question}
   A coin is tossed three times ,,consider the following events.A:'No head appears', B:'Exactly one head appears' and C:'Atleast two heads appears'. \\Do they form a set of mutually exclusive and exhaustive events?
\end{frame}


% Lists frame
\section{Lists in Beamer}
\begin{frame}{Solution}
The sample space of the experiment is \\

{S} = \myvec{HHH \\HHT\\HTH \\THH \\HTT \\ THT\\TTH \\ TTT}\\ 
and {A} = \myvec{TTT} ,\\
{B}=\myvec{HTT \\ THT \\ TTH} , \\
\end{frame} 
\begin{frame}

C =\myvec{HHT \\ HTH \\ THH \\ HHH} \\ 
Now ,\\ 
A $\cup$ {B} $\cup$ {C} = \myvec{HHH \\HHT\\HTH \\THH \\HTT \\ THT\\TTH \\ TTT} = {S} \\ 
Therefore , {A} ,{B} and {C} are exhaustive events .
Also ,\\ {A} $\cap$ {B} = $ \phi$ , {A} $\cap$ {C} = $\phi$ and {B} $\cap$ {C} = $\phi$\\

\end{frame}
\begin{frame}

Therefore , the events are pair-wise disjoint ,i.e, they are mutually exclusive.Hence ,A,B and C form a set of mutually exclusive and exhaustive events.


\end{frame}




\end{document}